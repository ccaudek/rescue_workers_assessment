% Options for packages loaded elsewhere
\PassOptionsToPackage{unicode}{hyperref}
\PassOptionsToPackage{hyphens}{url}
\PassOptionsToPackage{dvipsnames,svgnames,x11names}{xcolor}
%
\documentclass[
]{apa7}

\usepackage{amsmath,amssymb}
\usepackage{lmodern}
\usepackage{iftex}
\ifPDFTeX
  \usepackage[T1]{fontenc}
  \usepackage[utf8]{inputenc}
  \usepackage{textcomp} % provide euro and other symbols
\else % if luatex or xetex
  \usepackage{unicode-math}
  \defaultfontfeatures{Scale=MatchLowercase}
  \defaultfontfeatures[\rmfamily]{Ligatures=TeX,Scale=1}
\fi
% Use upquote if available, for straight quotes in verbatim environments
\IfFileExists{upquote.sty}{\usepackage{upquote}}{}
\IfFileExists{microtype.sty}{% use microtype if available
  \usepackage[]{microtype}
  \UseMicrotypeSet[protrusion]{basicmath} % disable protrusion for tt fonts
}{}
\makeatletter
\@ifundefined{KOMAClassName}{% if non-KOMA class
  \IfFileExists{parskip.sty}{%
    \usepackage{parskip}
  }{% else
    \setlength{\parindent}{0pt}
    \setlength{\parskip}{6pt plus 2pt minus 1pt}}
}{% if KOMA class
  \KOMAoptions{parskip=half}}
\makeatother
\usepackage{xcolor}
\setlength{\emergencystretch}{3em} % prevent overfull lines
\setcounter{secnumdepth}{-\maxdimen} % remove section numbering
% Make \paragraph and \subparagraph free-standing
\ifx\paragraph\undefined\else
  \let\oldparagraph\paragraph
  \renewcommand{\paragraph}[1]{\oldparagraph{#1}\mbox{}}
\fi
\ifx\subparagraph\undefined\else
  \let\oldsubparagraph\subparagraph
  \renewcommand{\subparagraph}[1]{\oldsubparagraph{#1}\mbox{}}
\fi


\providecommand{\tightlist}{%
  \setlength{\itemsep}{0pt}\setlength{\parskip}{0pt}}\usepackage{longtable,booktabs,array}
\usepackage{calc} % for calculating minipage widths
% Correct order of tables after \paragraph or \subparagraph
\usepackage{etoolbox}
\makeatletter
\patchcmd\longtable{\par}{\if@noskipsec\mbox{}\fi\par}{}{}
\makeatother
% Allow footnotes in longtable head/foot
\IfFileExists{footnotehyper.sty}{\usepackage{footnotehyper}}{\usepackage{footnote}}
\makesavenoteenv{longtable}
\usepackage{graphicx}
\makeatletter
\def\maxwidth{\ifdim\Gin@nat@width>\linewidth\linewidth\else\Gin@nat@width\fi}
\def\maxheight{\ifdim\Gin@nat@height>\textheight\textheight\else\Gin@nat@height\fi}
\makeatother
% Scale images if necessary, so that they will not overflow the page
% margins by default, and it is still possible to overwrite the defaults
% using explicit options in \includegraphics[width, height, ...]{}
\setkeys{Gin}{width=\maxwidth,height=\maxheight,keepaspectratio}
% Set default figure placement to htbp
\makeatletter
\def\fps@figure{htbp}
\makeatother
\newlength{\cslhangindent}
\setlength{\cslhangindent}{1.5em}
\newlength{\csllabelwidth}
\setlength{\csllabelwidth}{3em}
\newlength{\cslentryspacingunit} % times entry-spacing
\setlength{\cslentryspacingunit}{\parskip}
\newenvironment{CSLReferences}[2] % #1 hanging-ident, #2 entry spacing
 {% don't indent paragraphs
  \setlength{\parindent}{0pt}
  % turn on hanging indent if param 1 is 1
  \ifodd #1
  \let\oldpar\par
  \def\par{\hangindent=\cslhangindent\oldpar}
  \fi
  % set entry spacing
  \setlength{\parskip}{#2\cslentryspacingunit}
 }%
 {}
\usepackage{calc}
\newcommand{\CSLBlock}[1]{#1\hfill\break}
\newcommand{\CSLLeftMargin}[1]{\parbox[t]{\csllabelwidth}{#1}}
\newcommand{\CSLRightInline}[1]{\parbox[t]{\linewidth - \csllabelwidth}{#1}\break}
\newcommand{\CSLIndent}[1]{\hspace{\cslhangindent}#1}

\makeatletter
\makeatother
\makeatletter
\makeatother
\makeatletter
\@ifpackageloaded{caption}{}{\usepackage{caption}}
\AtBeginDocument{%
\ifdefined\contentsname
  \renewcommand*\contentsname{Table of contents}
\else
  \newcommand\contentsname{Table of contents}
\fi
\ifdefined\listfigurename
  \renewcommand*\listfigurename{List of Figures}
\else
  \newcommand\listfigurename{List of Figures}
\fi
\ifdefined\listtablename
  \renewcommand*\listtablename{List of Tables}
\else
  \newcommand\listtablename{List of Tables}
\fi
\ifdefined\figurename
  \renewcommand*\figurename{Figure}
\else
  \newcommand\figurename{Figure}
\fi
\ifdefined\tablename
  \renewcommand*\tablename{Table}
\else
  \newcommand\tablename{Table}
\fi
}
\@ifpackageloaded{float}{}{\usepackage{float}}
\floatstyle{ruled}
\@ifundefined{c@chapter}{\newfloat{codelisting}{h}{lop}}{\newfloat{codelisting}{h}{lop}[chapter]}
\floatname{codelisting}{Listing}
\newcommand*\listoflistings{\listof{codelisting}{List of Listings}}
\makeatother
\makeatletter
\@ifpackageloaded{caption}{}{\usepackage{caption}}
\@ifpackageloaded{subcaption}{}{\usepackage{subcaption}}
\makeatother
\makeatletter
\@ifpackageloaded{tcolorbox}{}{\usepackage[many]{tcolorbox}}
\makeatother
\makeatletter
\@ifundefined{shadecolor}{\definecolor{shadecolor}{rgb}{.97, .97, .97}}
\makeatother
\makeatletter
\makeatother
\ifLuaTeX
  \usepackage{selnolig}  % disable illegal ligatures
\fi
\IfFileExists{bookmark.sty}{\usepackage{bookmark}}{\usepackage{hyperref}}
\IfFileExists{xurl.sty}{\usepackage{xurl}}{} % add URL line breaks if available
\urlstyle{same} % disable monospaced font for URLs
\hypersetup{
  pdftitle={Exploring the Role of Self-Compassion in Promoting Resilience and Well-Being Among Rescue Workers},
  colorlinks=true,
  linkcolor={blue},
  filecolor={Maroon},
  citecolor={Blue},
  urlcolor={Blue},
  pdfcreator={LaTeX via pandoc}}

\title{Exploring the Role of Self-Compassion in Promoting Resilience and
Well-Being Among Rescue Workers}
\author{}
\date{}

\begin{document}
\maketitle
\ifdefined\Shaded\renewenvironment{Shaded}{\begin{tcolorbox}[interior hidden, enhanced, breakable, borderline west={3pt}{0pt}{shadecolor}, sharp corners, boxrule=0pt, frame hidden]}{\end{tcolorbox}}\fi

prova uno due tre (Mao et al., 2022)

\hypertarget{instruments}{%
\subsection{Instruments}\label{instruments}}

Beside the specific questions for RWs, we administered the following
scales to both groups.

\textbf{Self-Compassion Scale.} The Self-Compassion Scale {[}SCS; Neff
(2003); Italian version by Veneziani et al. (2017){]} is a 26 items
self-report questionnaire, was used to evaluate self-compassion. High
levels of self-compassion reflect an ability to be kind and
understanding toward oneself, even in difficult times. Three sub-sets of
items are considered indicators of compassionate self-responding (CS):
SK, CH, and MI. The remaining three sub-sets of items are considered
indicators of uncompassionate self-responding (UCS): SJ, IS, and OI. The
SCS total score (SCS-TS) is computed by averaging the mean scores of the
positive subscales and of the reversely-scored negative subscales. In
the present sample, total reliability was \(\alpha\) = .89, \(\omega\) =
.92. Reliability was adequate for the subscales. SK: \(\alpha\) = .84,
\(\omega\) = .90; CH: \(\alpha\) = .72, \(\omega\) = .78; MI: \(\alpha\)
= .75, \(\omega\) = .78; SJ: \(\alpha\) = .84, \(\omega\) = .85; IS:
\(\alpha\) = .86, \(\omega\) = .89; OI: \(\alpha\) = .83, \(\omega\) =
.86. Further psychometric properties of the SCS are presented in the
Results section.

\textbf{Post-Traumatic Growth Inventory.} The Post-Traumatic Growth
Inventory {[}PTGI; Tedeschi \& Calhoun (1996){]} is a 21 item
self-report measure. PTGI evaluates the growth following one or more
stressful or traumatic events in one's life. The PTGI comprises five
sub-scales (Relating to others, New possibilities, Personal strength,
Appreciation of life, and Spiritual change) and has good internal
consistency, construct-convergent validity, and discriminant validity
(Tedeschi \& Calhoun, 1996). Also the Italian version has good internal
consistency and validity (Prati \& Pietrantoni, 2014). In the present
sample, total reliability was \(\alpha\) = .95, \(\omega\) = .95.
Reliability was adequate for the subscales. Relating to others:
\(\alpha\) = .91, \(\omega\) = .92; New possibilities: \(\alpha\) = .86,
\(\omega\) = .89; Personal strength: \(\alpha\) = .82, \(\omega\) = .83;
Appreciation of life: \(\alpha\) = .79, \(\omega\) = .81; Spiritual
changes: \(\alpha\) = .74, \(\omega\) = .74.

\textbf{Impact of Event Scale - Revised.} The Impact of Event Scale -
Revised {[}IES-R; Weiss (2007){]} is a 22-item self-report measure
assessing subjective distress caused by traumatic events and it is based
on the views of the core phenomena of traumatic-stress reactions:
intrusion (B criteria in the DSM-IV PTSD diagnosis), avoidance (C
criteria), and persistent hyper-arousal. Correspondingly, the IES-R
comprises the sub-scales of Intrusion, Avoidance, and Hyperarousal. The
IES-R is widely used to assess the symptomatology of the PTSD in rescue
workers. The IES-R show good internal consistency and test--retest
stability. The Italian translation show good psychometric properties,
good concurrent and discriminant validity, and good test--retest
reliability (Craparo et al., 2013). In the present sample, total
reliability was \(\alpha\) = .93, \(\omega\) = .94. Reliability was also
high for the subscales. Intrusion: \(\alpha\) = .90, \(\omega\) = .91;
Avoidance: \(\alpha\) = .78, \(\omega\) = .82; Hyperarousal: \(\alpha\)
= .85, \(\omega\) = .87.

\textbf{Coping Orientation to Problems Experienced} The Coping
Orientation to Problems Experienced test (COPE; Carver et al. (1989){]}
is a self-report questionnaire used to evaluate the skills and
strategies adopted to face stressful and difficult events. The COPE
comprises five dimensions (social support, avoidance strategies,
positive attitude, problem-solving, and transcendent orientation) and
has adequate internal consistency, and convergent and discriminant
validity (Carver et al., 1989). Factor analyses on the Italian version
of the COPE (Sica et al., 1997, 2008, 2021) have demonstrated that the
scales can be grouped into the following dimensions: Problem-focused,
Social Support, Avoidance-oriented, Positive-oriented, and
Transcendent-oriented. In the present study, we were interested in the
adaptive coping strategies that can be considered on par with
self-compassion. Therefore, we only administered the Positive-oriented
and Problem-focused subscales. In fact, it is generally acknowledged
that positive-oriented coping (e.g., positive reinterpretation and
growth, acceptance) is beneficial for managing difficult situations
(e.g., Sica et al., 2021). Moreover, problem-focused coping is an
adaptive and active coping approach (e.g., accepting social support,
solving problem, seeking advice or information, and analyzing situation
logically) that has been shown to be effective in mediating the
development of PTG from self-compassion (Munroe et al., 2022). In the
present sample, total reliability was \(\alpha\) = .86, \(\omega\) =
.87. Reliability was also adequate for the subscales. Positive-oriented:
\(\alpha\) = .75, \(\omega\) = .77; Problem-focused: \(\alpha\) = .84,
\(\omega\) = .85.

\textbf{NEO-Five Factor Inventory.} The NEO-Five Factor Inventory
(NEO-FFI-60; Costa \& McCrae (1992)) is a 60 items self-report
questionnaire, was used to assess five broad domains of personality:
Neuroticism (N), Extraversion (E), Openness to experience (O),
Agreeableness (A), and Conscientiousness (C). The internal consistency
of the five sub-scales of the NEO-FFI-60 is adequate (Murray et al.,
2003). Items from the Neuroticism and Extraversion domains served as an
indicator of latent variables in the structural model; scores for the
other three domains were employed for group comparisons in descriptive
analyses. In the present sample, we used the Italian version by Caprara
et al. (2001). The Neuroticism (\(\alpha\) = .88, \(\omega\) = .89),
Extraversion (\(\alpha\) = .74, \(\omega\) = .77) and Conscientiousness
(\(\alpha\) = .78, \(\omega\) = .82) subscales showed adequate internal
consistency, whereas reliability was low for the Agreableness
(\(\alpha\) = .59, \(\omega\) = .64) and Openness (\(\alpha\) = .59,
\(\omega\) = .60) subscales.

\hypertarget{statistical-analyses}{%
\subsection{Statistical analyses}\label{statistical-analyses}}

Latent Profile Analysis (LPA) is a finite mixture modeling technique
that partitions individuals into discrete classes based on their
responses to observed variables. This technique is particularly useful
for identifying subgroups of individuals that can be meaningfully
compared (Lanza \& Rhoades, 2013). The primary objectives of LVA are
twofold. Firstly, to ensure homogeneity within each identified profile
so that individuals grouped together are as similar as possible.
Secondly, to maximize heterogeneity between profiles so that each
profile accurately represents a distinct grouping of individuals. The
classes generated by LVA are considered latent since they are not
directly observable but are inferred based on similarities in the data.
LPA accounts for measurement errors related to the uncertainty in
profile membership and provides fit statistics to determine the number
of profiles that best represent the data.

The purpose of the LPA was to detect distinct subgroups of RWs who have
different profiles on personality dimensions, protective factors, and
outcome variables. Standardized scores for five personality measures
(neuroticism, extraversion, openness, agreeableness, conscientiousness),
three dimensions of coping (COPE-active coping, COPE-avoidance coping,
COPE-social emotional coping), perceived social support (MSPSS), and
post-traumatic stress (measured using the IES-R) from the RWs were used
as observed indicators for the LPA.

We fitted a series of LPA models, ranging from 1 to 10 profiles, using
1000 sets of starting values. To determine the optimal number of
profiles, we used information criteria, including Bayesian information
criterion (BIC), Akaike information criterion (AIC), and adjusted BIC.
We selected the model with the lowest value of these criteria,
indicating a better fit. Additionally, we evaluated the accuracy of the
classification of individuals into the appropriate profile using
entropy, with values closer to 1 indicating higher separation among
classes (\textgreater{} 0.80 represents high separation). We also
employed the Lo-Mendell-Rubin likelihood ratio test (LMR-LRT), a test
statistic to compare the fit of a model with a lower versus higher
number of profiles. We used MPLUS 8.6 and the \texttt{R} software for
all statistical analyses.

\hypertarget{references}{%
\subsection*{References}\label{references}}
\addcontentsline{toc}{subsection}{References}

\hypertarget{refs}{}
\begin{CSLReferences}{1}{0}
\leavevmode\vadjust pre{\hypertarget{ref-caprara2001brand}{}}%
Caprara, G. V., Barbaranelli, C., \& Guido, G. (2001). Brand
personality: How to make the metaphor fit? \emph{Journal of Economic
Psychology}, \emph{22}(3), 377--395.

\leavevmode\vadjust pre{\hypertarget{ref-carver1989assessing}{}}%
Carver, C. S., Scheier, M. F., \& Weintraub, J. K. (1989). Assessing
coping strategies: A theoretically based approach. \emph{Journal of
Personality and Social Psychology}, \emph{56}(2), 267--283.

\leavevmode\vadjust pre{\hypertarget{ref-costa1992normal}{}}%
Costa, P. T., \& McCrae, R. R. (1992). Normal personality assessment in
clinical practice: The NEO personality inventory. \emph{Psychological
Assessment}, \emph{4}(1), 5--13.

\leavevmode\vadjust pre{\hypertarget{ref-craparo2013impact}{}}%
Craparo, G., Faraci, P., Rotondo, G., \& Gori, A. (2013). The impact of
event scale--revised: Psychometric properties of the italian version in
a sample of flood victims. \emph{Neuropsychiatric Disease and
Treatment}, \emph{9}, 1427--1432.

\leavevmode\vadjust pre{\hypertarget{ref-lanza2013latent}{}}%
Lanza, S. T., \& Rhoades, B. L. (2013). Latent class analysis: An
alternative perspective on subgroup analysis in prevention and
treatment. \emph{Prevention Science}, \emph{14}(2), 157--168.

\leavevmode\vadjust pre{\hypertarget{ref-mao2022concept}{}}%
Mao, X., Hu, X., \& Loke, A. Y. (2022). A concept analysis on disaster
resilience in rescue workers: The psychological perspective.
\emph{Disaster Medicine and Public Health Preparedness}, \emph{16}(4),
1682--1691.

\leavevmode\vadjust pre{\hypertarget{ref-munroe2022using}{}}%
Munroe, M., Al-Refae, M., Chan, H. W., \& Ferrari, M. (2022). Using
self-compassion to grow in the face of trauma: The role of positive
reframing and problem-focused coping strategies. \emph{Psychological
Trauma: Theory, Research, Practice, and Policy}, \emph{14}(S1), S157.

\leavevmode\vadjust pre{\hypertarget{ref-murray2003neo}{}}%
Murray, G., Rawlings, D., Allen, N. B., \& Trinder, J. (2003). NEO
five-factor inventory scores: Psychometric properties in a community
sample. \emph{Measurement and Evaluation in Counseling and Development},
\emph{36}(3), 140--149.

\leavevmode\vadjust pre{\hypertarget{ref-neff2003self}{}}%
Neff, K. D. (2003). Self-compassion: An alternative conceptualization of
a healthy attitude toward oneself. \emph{Self and Identity},
\emph{2}(2), 85--101.

\leavevmode\vadjust pre{\hypertarget{ref-prati2014italian}{}}%
Prati, G., \& Pietrantoni, L. (2014). Italian adaptation and
confirmatory factor analysis of the full and the short form of the
posttraumatic growth inventory. \emph{Journal of Loss and Trauma},
\emph{19}(1), 12--22.

\leavevmode\vadjust pre{\hypertarget{ref-sica2021facing}{}}%
Sica, C., Latzman, R. D., Caudek, C., Cerea, S., Colpizzi, I., Caruso,
M., Giulini, P., \& Bottesi, G. (2021). Facing distress in coronavirus
era: The role of maladaptive personality traits and coping strategies.
\emph{Personality and Individual Differences}, \emph{177}, 110833.

\leavevmode\vadjust pre{\hypertarget{ref-sica2008coping}{}}%
Sica, C., Magni, C., Ghisi, M., Altoè, G., Sighinolfi, C., Chiri, L. R.,
\& Franceschini, S. (2008). Coping orientation to problems
experienced-nuova versione italiana (COPE-NVI): Uno strumento per la
misura degli stili di coping. \emph{Psicoterapia Cognitiva e
Comportamentale}, \emph{14}(1), 27.

\leavevmode\vadjust pre{\hypertarget{ref-sica1997coping}{}}%
Sica, C., Novara, C., Dorz, S., \& Sanavio, E. (1997). Coping
strategies: Evidence for cross-cultural differences? A preliminary study
with the italian version of coping orientations to problems experienced
(COPE). \emph{Personality and Individual Differences}, \emph{23}(6),
1025--1029.

\leavevmode\vadjust pre{\hypertarget{ref-tedeschi1996posttraumatic}{}}%
Tedeschi, R. G., \& Calhoun, L. G. (1996). The posttraumatic growth
inventory: Measuring the positive legacy of trauma. \emph{Journal of
Traumatic Stress}, \emph{9}(3), 455--471.

\leavevmode\vadjust pre{\hypertarget{ref-veneziani2017self}{}}%
Veneziani, C. A., Fuochi, G., \& Voci, A. (2017). Self-compassion as a
healthy attitude toward the self: Factorial and construct validity in an
italian sample. \emph{Personality and Individual Differences},
\emph{119}, 60--68.

\leavevmode\vadjust pre{\hypertarget{ref-weiss2007impact}{}}%
Weiss, D. S. (2007). The impact of event scale: revised. In
\emph{Cross-cultural assessment of psychological trauma and PTSD} (pp.
219--238). Springer.

\end{CSLReferences}



\end{document}
