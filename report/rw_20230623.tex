% Options for packages loaded elsewhere
\PassOptionsToPackage{unicode}{hyperref}
\PassOptionsToPackage{hyphens}{url}
%
\documentclass[
  man]{apa7}
\usepackage{amsmath,amssymb}
\usepackage{iftex}
\ifPDFTeX
  \usepackage[T1]{fontenc}
  \usepackage[utf8]{inputenc}
  \usepackage{textcomp} % provide euro and other symbols
\else % if luatex or xetex
  \usepackage{unicode-math} % this also loads fontspec
  \defaultfontfeatures{Scale=MatchLowercase}
  \defaultfontfeatures[\rmfamily]{Ligatures=TeX,Scale=1}
\fi
\usepackage{lmodern}
\ifPDFTeX\else
  % xetex/luatex font selection
\fi
% Use upquote if available, for straight quotes in verbatim environments
\IfFileExists{upquote.sty}{\usepackage{upquote}}{}
\IfFileExists{microtype.sty}{% use microtype if available
  \usepackage[]{microtype}
  \UseMicrotypeSet[protrusion]{basicmath} % disable protrusion for tt fonts
}{}
\makeatletter
\@ifundefined{KOMAClassName}{% if non-KOMA class
  \IfFileExists{parskip.sty}{%
    \usepackage{parskip}
  }{% else
    \setlength{\parindent}{0pt}
    \setlength{\parskip}{6pt plus 2pt minus 1pt}}
}{% if KOMA class
  \KOMAoptions{parskip=half}}
\makeatother
\usepackage{xcolor}
\usepackage{graphicx}
\makeatletter
\def\maxwidth{\ifdim\Gin@nat@width>\linewidth\linewidth\else\Gin@nat@width\fi}
\def\maxheight{\ifdim\Gin@nat@height>\textheight\textheight\else\Gin@nat@height\fi}
\makeatother
% Scale images if necessary, so that they will not overflow the page
% margins by default, and it is still possible to overwrite the defaults
% using explicit options in \includegraphics[width, height, ...]{}
\setkeys{Gin}{width=\maxwidth,height=\maxheight,keepaspectratio}
% Set default figure placement to htbp
\makeatletter
\def\fps@figure{htbp}
\makeatother
\setlength{\emergencystretch}{3em} % prevent overfull lines
\providecommand{\tightlist}{%
  \setlength{\itemsep}{0pt}\setlength{\parskip}{0pt}}
\setcounter{secnumdepth}{-\maxdimen} % remove section numbering
% Make \paragraph and \subparagraph free-standing
\ifx\paragraph\undefined\else
  \let\oldparagraph\paragraph
  \renewcommand{\paragraph}[1]{\oldparagraph{#1}\mbox{}}
\fi
\ifx\subparagraph\undefined\else
  \let\oldsubparagraph\subparagraph
  \renewcommand{\subparagraph}[1]{\oldsubparagraph{#1}\mbox{}}
\fi
\newlength{\cslhangindent}
\setlength{\cslhangindent}{1.5em}
\newlength{\csllabelwidth}
\setlength{\csllabelwidth}{3em}
\newlength{\cslentryspacingunit} % times entry-spacing
\setlength{\cslentryspacingunit}{\parskip}
\newenvironment{CSLReferences}[2] % #1 hanging-ident, #2 entry spacing
 {% don't indent paragraphs
  \setlength{\parindent}{0pt}
  % turn on hanging indent if param 1 is 1
  \ifodd #1
  \let\oldpar\par
  \def\par{\hangindent=\cslhangindent\oldpar}
  \fi
  % set entry spacing
  \setlength{\parskip}{#2\cslentryspacingunit}
 }%
 {}
\usepackage{calc}
\newcommand{\CSLBlock}[1]{#1\hfill\break}
\newcommand{\CSLLeftMargin}[1]{\parbox[t]{\csllabelwidth}{#1}}
\newcommand{\CSLRightInline}[1]{\parbox[t]{\linewidth - \csllabelwidth}{#1}\break}
\newcommand{\CSLIndent}[1]{\hspace{\cslhangindent}#1}
\ifLuaTeX
\usepackage[bidi=basic]{babel}
\else
\usepackage[bidi=default]{babel}
\fi
\babelprovide[main,import]{english}
% get rid of language-specific shorthands (see #6817):
\let\LanguageShortHands\languageshorthands
\def\languageshorthands#1{}
% Manuscript styling
\usepackage{upgreek}
\captionsetup{font=singlespacing,justification=justified}

% Table formatting
\usepackage{longtable}
\usepackage{lscape}
% \usepackage[counterclockwise]{rotating}   % Landscape page setup for large tables
\usepackage{multirow}		% Table styling
\usepackage{tabularx}		% Control Column width
\usepackage[flushleft]{threeparttable}	% Allows for three part tables with a specified notes section
\usepackage{threeparttablex}            % Lets threeparttable work with longtable

% Create new environments so endfloat can handle them
% \newenvironment{ltable}
%   {\begin{landscape}\centering\begin{threeparttable}}
%   {\end{threeparttable}\end{landscape}}
\newenvironment{lltable}{\begin{landscape}\centering\begin{ThreePartTable}}{\end{ThreePartTable}\end{landscape}}

% Enables adjusting longtable caption width to table width
% Solution found at http://golatex.de/longtable-mit-caption-so-breit-wie-die-tabelle-t15767.html
\makeatletter
\newcommand\LastLTentrywidth{1em}
\newlength\longtablewidth
\setlength{\longtablewidth}{1in}
\newcommand{\getlongtablewidth}{\begingroup \ifcsname LT@\roman{LT@tables}\endcsname \global\longtablewidth=0pt \renewcommand{\LT@entry}[2]{\global\advance\longtablewidth by ##2\relax\gdef\LastLTentrywidth{##2}}\@nameuse{LT@\roman{LT@tables}} \fi \endgroup}

% \setlength{\parindent}{0.5in}
% \setlength{\parskip}{0pt plus 0pt minus 0pt}

% Overwrite redefinition of paragraph and subparagraph by the default LaTeX template
% See https://github.com/crsh/papaja/issues/292
\makeatletter
\renewcommand{\paragraph}{\@startsection{paragraph}{4}{\parindent}%
  {0\baselineskip \@plus 0.2ex \@minus 0.2ex}%
  {-1em}%
  {\normalfont\normalsize\bfseries\itshape\typesectitle}}

\renewcommand{\subparagraph}[1]{\@startsection{subparagraph}{5}{1em}%
  {0\baselineskip \@plus 0.2ex \@minus 0.2ex}%
  {-\z@\relax}%
  {\normalfont\normalsize\itshape\hspace{\parindent}{#1}\textit{\addperi}}{\relax}}
\makeatother

% \usepackage{etoolbox}
\makeatletter
\patchcmd{\HyOrg@maketitle}
  {\section{\normalfont\normalsize\abstractname}}
  {\section*{\normalfont\normalsize\abstractname}}
  {}{\typeout{Failed to patch abstract.}}
\patchcmd{\HyOrg@maketitle}
  {\section{\protect\normalfont{\@title}}}
  {\section*{\protect\normalfont{\@title}}}
  {}{\typeout{Failed to patch title.}}
\makeatother

\usepackage{xpatch}
\makeatletter
\xapptocmd\appendix
  {\xapptocmd\section
    {\addcontentsline{toc}{section}{\appendixname\ifoneappendix\else~\theappendix\fi\\: #1}}
    {}{\InnerPatchFailed}%
  }
{}{\PatchFailed}
\keywords{keywords\newline\indent Word count: X}
\DeclareDelayedFloatFlavor{ThreePartTable}{table}
\DeclareDelayedFloatFlavor{lltable}{table}
\DeclareDelayedFloatFlavor*{longtable}{table}
\makeatletter
\renewcommand{\efloat@iwrite}[1]{\immediate\expandafter\protected@write\csname efloat@post#1\endcsname{}}
\makeatother
\usepackage{lineno}

\linenumbers
\usepackage{csquotes}
\ifLuaTeX
  \usepackage{selnolig}  % disable illegal ligatures
\fi
\IfFileExists{bookmark.sty}{\usepackage{bookmark}}{\usepackage{hyperref}}
\IfFileExists{xurl.sty}{\usepackage{xurl}}{} % add URL line breaks if available
\urlstyle{same}
\hypersetup{
  pdftitle={Investigating the Role of Self-Compassion in Promoting Resilience and Reducing Negative Outcomes Among Rescue Workers},
  pdfauthor={Corrado Caudek1, Claudio Sica2, Celeste Berti1, Ilaria Colpizzi1, Virginia Alfei1, \& Diletta Bardazzi1},
  pdflang={en-EN},
  pdfkeywords={keywords},
  hidelinks,
  pdfcreator={LaTeX via pandoc}}

\title{Investigating the Role of Self-Compassion in Promoting Resilience and Reducing Negative Outcomes Among Rescue Workers}
\author{Corrado Caudek\textsuperscript{1}, Claudio Sica\textsuperscript{2}, Celeste Berti\textsuperscript{1}, Ilaria Colpizzi\textsuperscript{1}, Virginia Alfei\textsuperscript{1}, \& Diletta Bardazzi\textsuperscript{1}}
\date{}


\shorttitle{Self-Compassion and Resilience}

\authornote{

Add complete departmental affiliations for each author here. Each new line herein must be indented, like this line.

Enter author note here.

The authors made the following contributions. Corrado Caudek: Conceptualization, Writing - Original Draft Preparation, Writing - Review \& Editing.

Correspondence concerning this article should be addressed to Corrado Caudek, NEUROFARBA Department, Psychology Section, University of Firenze, Italy. E-mail: \href{mailto:corrado.caudek@unifi.it}{\nolinkurl{corrado.caudek@unifi.it}}

}

\affiliation{\vspace{0.5cm}\textsuperscript{1} NEUROFARBA Department, Psychology Section, University of Florence, Italy\\\textsuperscript{2} Health Sciences Department, Psychology Section, University of Florence, Italy}

\abstract{%
This study investigates the relationship between self-compassion and latent profiles of risk and protective factors for rescue workers. Using a cross-sectional design, data were collected from a sample of 782 rescue workers and 338 controls using a survey instrument that assessed levels of self-compassion and various risk and protective factors, including coping styles, perceived social support, personality dimensions, post-traumatic stress, and post-traumatic growth. Latent profile analysis revealed three distinct profiles of risk and protective factors: protective, protective under duress, and disfunctional. Results showed that the self-judgment, over-identification, and isolation components of self-compassion were negatively associated with higher levels of protective factors and lower levels of risk factors. Moreover, self-compassion was found to be a stronger predictor of resilience and well-being than any other risk or protective factor profile. These findings suggest that self-compassion may be a critical resource for promoting resilience and well-being among rescue workers, even in the presence of high-risk work environments. Implications for future research and practical applications are discussed.
}



\begin{document}
\maketitle

Rescue workers (RWs) and healthcare professionals often experience trauma and suffering, which can result in ``vicarious traumatization'' (McCann \& Pearlman, 1990) and negative mental health outcomes. Self-compassion (Neff, 2003), a way of relating to oneself, has been recognized as a potential protective factor against psychopathology and has positive effects on mental health.

The primary objective of this study is to examine the relationship between individual differences in personality traits, protective factors, risk factors, and outcomes among RWs and their level of self-compassion. By employing Latent Profile Analysis, the study aims to identify distinct profiles of individual differences. Specifically, it seeks to determine whether individuals in an adaptive profile exhibit higher levels of positive self-compassion and lower levels of negative self-compassion, while those in a maladaptive profile demonstrate the opposite pattern. By investigating the diverse profiles of individual differences among RWs, this study aims to gain valuable insights into the factors that promote or impede the cultivation of self-compassion. These findings can inform interventions and strategies aimed at enhancing the well-being and resilience of RWs.

\hypertarget{vicarious-traumatization-and-self-compassion}{%
\subsection{Vicarious traumatization and self-compassion}\label{vicarious-traumatization-and-self-compassion}}

RWs and healthcare professionals often experience distress and negative mental health outcomes as a result of their exposure to the suffering of others. This includes the development of compassion fatigue (Joinson, 1992), burnout (Chatzea, Sifaki-Pistolla, Vlachaki, Melidoniotis, \& Pistolla, 2018), and even Post-Traumatic Stress Disorder {[}PTSD; Tahernejad et al. (2023){]}. The concept of ``vicarious traumatization'' (Figley, 1995; McCann \& Pearlman, 1990) recognizes that these professionals, through their repeated and close interactions with trauma victims, can experience emotional distress as indirect victims of the same trauma. Those in helping professions, including RWs and healthcare workers, are particularly susceptible to vicarious traumatization due to their involvement in more stressful situations (Argentero \& Setti, 2011). Therefore, understanding and promoting factors that contribute to their well-being is crucial in mitigating the negative impact of vicarious traumatization and supporting their resilience (Mao, Hu, \& Loke, 2022).

In recent years, the concept of ``self'' has gained prominence in understanding individual differences in coping with stress (Beck, Steer, Epstein, \& Brown, 1990). Within the realm of self, self-compassion, which involves specific ways of relating to oneself, has shown to have a positive impact on mental health (MacBeth \& Gumley, 2012). Cultivating a compassionate mindset towards oneself may also serve as a protective factor against psychopathology, including PTSD (Wilson, Mackintosh, Power, \& Chan, 2019; Wong \& Yeung, 2017). Therefore, integrating the notion of self-compassion alongside the previously discussed concepts holds great potential for enhancing the well-being and resilience of RWs and healthcare professionals (Hashem \& Zeinoun, 2020).

Self-compassion is typically measured using the Self-Compassion Scale (SCS). The SCS measures six dimensions of self-compassion, three of which evaluate the active components of self-compassion. These dimensions include Self-kindness (SK), Common humanity (CH), and Mindfulness (MI), which involve being kind and understanding towards oneself, recognizing that personal failures and pain are common experiences, and maintaining awareness of one's painful thoughts and feelings. The remaining three dimensions evaluate the ``hindrances'' to self-compassion, including Self-judgment (SJ), Isolation (IS), and Overidentification (OI). These dimensions assess factors that hinder self-compassion, such as being self-critical and unsympathetic towards one's shortcomings, isolating oneself from others, and over-identifying with one's painful thoughts and emotions (Neff, 2022).

The relationship between self-compassion (SC) and rescue workers (RW) poses two unresolved questions in the literature. Firstly, it is unclear if SC functions as a protective factor in RW similar to the general population. If there is a deficiency in this protective factor among RWs, it is unknown if all aspects of self-compassion are equally affected. Secondly, considering individual variations in SC levels, it is unclear if consistent characteristics exist among RWs that differentiate those who rely more or less on SC as a protective factor. This study aims to address these questions by utilizing Latent Profile Analysis (A. Liu, Wang, \& Wu, 2021; Ullrich-French \& Cox, 2020) to identify distinct profiles of personality, protective factors, risk factors, and outcomes among RWs. We will then examine if individuals in the adaptive profile exhibit higher positive dimensions of SC and lower negative dimensions, while those in the maladaptive profile show the opposite pattern.

\hypertarget{personality-traits-as-protective-factors}{%
\subsection{Personality traits as protective factors}\label{personality-traits-as-protective-factors}}

Personality traits play a crucial role in determining an individual's susceptibility to burnout and are strongly associated with anxiety and depressive disorders. Research has shown that high levels of neuroticism and low scores in the other four Big Five personality traits are more likely to experience emotional disorders (Bienvenu et al., 2004; Karsten et al., 2012; Kotov, Gamez, Schmidt, \& Watson, 2010).

A meta-analysis conducted by Malouff, Thorsteinsson, and Schutte (2005) revealed revealed a specific pattern of personality traits associated with mood disorders. Individuals with mood disorders tend to have higher scores in neuroticism and lower scores in extraversion, conscientiousness, and agreeableness, with the strongest effects observed for neuroticism, extraversion, and conscientiousness. However, no significant association was found between mood disorders and openness. Another meta-analysis by Swider and Zimmerman (2010) showed that the Five Factor Model of Personality, including neuroticism, extraversion, agreeableness, conscientiousness, and openness, collectively explains a significant proportion of the variance in job burnout levels. Specifically, neuroticism has been identified as the strongest correlate of burnout (Bianchi, 2018). Individuals with low extraversion levels tend to focus on negative aspects of events and primarily employ emotion-focused coping strategies (Connor-Smith \& Flachsbart, 2007). Moreover, individuals with low conscientiousness levels are more likely to experience depersonalization and reduced personal accomplishment (Kokkinos, 2007). Conversely, individuals with high agreeableness levels demonstrate better interpersonal relationships at work, characterized by traits such as gentleness and cooperation, which are associated with lower burnout levels (Angelini, 2023). However, the literature has not established a clear link between openness and burnout (Angelini, 2023; Răducu \& Stănculescu, 2022). Thus, it can be inferred that elevated levels of neuroticism and lower levels of extraversion, agreeableness, and conscientiousness (excluding openness) may serve as a ``personality marker'' for rescue workers who may have difficulty mobilizing internal resources and building resistance against stressors.

\hypertarget{coping-strategies}{%
\subsection{Coping strategies}\label{coping-strategies}}

Coping strategies are closely tied to personality traits (Sica et al., 2021). Maladaptive coping strategies, including suppression, rumination, and avoidance, consistently relate to negative psychological well-being (Joormann et al., 2016; Liu et al., 2017; Moritz et al., 2016). In contrast, the absence of adaptive coping strategies appears less relevant for the development of psychological disorders (Aldao et al., 2012; Moritz et al., 2016).

A meta-analysis by Connor-Smith et al.~(2007) highlights the associations between extraversion, neuroticism, agreeableness, openness, conscientiousness, and coping strategies. Extraversion is linked to problem-focused and emotion-focused strategies, while neuroticism relates to emotion-focused and avoidance-oriented strategies. Agreeableness and openness show weak associations with coping, particularly in relation to social support and problem-focused strategies. Conscientiousness strongly correlates with problem-focused strategies. Avoidance-oriented strategies, such as substance use, are negatively associated with agreeableness and conscientiousness (Afshar et al., 2015; Connor-Smith et al., 2007).

\hypertarget{social-support}{%
\subsection{Social support}\label{social-support}}

Meta-analyses, such as the study conducted by Berger et al.~(2012), consistently reveal a higher prevalence of PTSD among RWs compared to the general population. Effective coping and resilience in RWs rely on access to internal and external resources. Notably, perceived social support from colleagues and superiors plays a significant role in mitigating burnout among RWs by reducing emotional exhaustion, depersonalization, and inefficacy (Setti et al., 2016). These findings align with previous research indicating a link between social support, lower levels of burnout, and posttraumatic symptoms (Armstrong et al., 2004). The stress-buffering hypothesis (Cohen \& Wills, 1985), the social support deterioration model (Norris et al., 1996), and the conservation of resources model (Hobfoll, 1989) all propose that perceived social support acts as a protective factor, shielding individuals from the adverse effects of stress.

\hypertarget{purpose-of-the-study}{%
\subsection{Purpose of the study}\label{purpose-of-the-study}}

The purpose of this study is to identify distinct resilience profiles among RWs. We anticipate that a lower-resilience profile will exhibit elevated levels of neuroticism, maladaptive coping strategies, decreased extraversion, agreeableness, conscientiousness, and perceived social support, as well as higher levels of reported post-traumatic symptoms. Conversely, a higher-resilience profile is expected to demonstrate the opposite pattern of characteristics. A critical hypothesis to be tested is that the ``low resilience'' profile will be associated with higher levels of ``negative'' self-compassion and lower levels of ``positive'' self-compassion when compared to the ``high resilience'' profile group.

To investigate the potential deficit in self-compassion among the ``low resilience'' profile, we will compare the overall level of self-compassion between RWs and a control community sample. Furthermore, we will examine the impact of job qualifications, specifically the level of direct versus indirect involvement in alleviating others' suffering, on the reliance on self-compassion.

\hypertarget{methods}{%
\section{Methods}\label{methods}}

\hypertarget{instruments}{%
\subsubsection{Instruments}\label{instruments}}

In addition to administering specific questions tailored for the RW group, we also utilized the following scales to assess both participant groups.

\emph{Self-Compassion}. The Self-Compassion Scale {[}SCS; Neff (2003){]} was used to measure self-compassione. The SCS is a 26-item self-report measure designed to assess self-compassion, or the ability to extend kindness and understanding to oneself during challenging times. The SCS is composed of six subscales, with three of them (Self-Kindness, SK; Common-Humanity, CH; and Mindfulness, MI) measuring compassionate self-responding, and the remaining three (Self-Judgment, SJ; Isolation, IS; and Over-Identification, OI) assessing uncompassionate self-responding. The SCS total score (SCS-TS) is obtained by inverting the scores of the subscales related to uncompassionate self-responding. The Italian version of the SCS by Veneziani, Fuochi, and Voci (2017) was used in the present study. The SCS demonstrated good internal consistency, with a total reliability of \(\omega\) = .92. The subscales also demonstrated adequate reliability: SK (\(\omega\) = .90), CH (\(\omega\) = .78), MI (\(\omega\) = .78), SJ (\(\omega\) = .85), IS (\(\omega\) = .89), and OI (\(\omega\) = .86).

\emph{Personality traits}. The NEO-Five Factor Inventory (NEO-FFI-60; Costa and McCrae (1992)) was employed to examine personality traits. The NEO-FFI-60 is a widely used 60-item self-report questionnaire that assesses five broad domains of personality: Neuroticism (N), Extraversion (E), Openness to experience (O), Agreeableness (A), and Conscientiousness (C). The internal consistency of the five sub-scales of the NEO-FFI-60 has been found to be adequate (Murray, Rawlings, Allen, \& Trinder, 2003). In the current study, we used the Italian version of the NEO-FFI-60 developed by Caprara, Barbaranelli, and Guido (2001). The Neuroticism (\(\omega\) = .92), Extraversion (\(\omega\) = .83), Conscientiousness (\(\omega\) = .87), and Openness (\(\omega\) = .78) subscales showed adequate internal consistency, whereas reliability was low for Agreableness (\(\omega\) = .66) -- see also Burton, Delvecchio, Germani, and Mazzeschi (2021).

\emph{Adaptive and maladaptive coping strategies}. The Coping Orientation to Problems Experienced (COPE) test was used to assess adaptive and maladaptive coping. The COPE test (Carver, Scheier, \& Weintraub, 1989) is a self-report questionnaire commonly used to assess an individual's coping skills and strategies when dealing with stressful and challenging events. In the present study, we utilized the scoring system proposed by Lyne and Roger (2000), which divides the COPE items into three subscales: Active Coping, Emotion-Focused Coping, and Avoidance Coping. Active Coping reflects a constructive and active approach to coping, in which individuals acknowledge the occurrence of a stressful situation and take action to address the problem through problem-solving, gathering information, and analyzing the situation logically. The other two subscales, Emotion-Focused Coping (expressing feelings and seeking emotional support) and Avoidance Coping (behavioural disengagement (giving up), denial, and mental disengagement), represent more passive approaches to problem-solving, suggesting a belief that the situation cannot be changed. These subscales assess maladaptive coping strategies. In our study, we utilized the Italian version of the COPE questionnaire developed by Sica, Novara, Dorz, and Sanavio (1997; Sica et al., 2021; see also Sica et al., 2008). The reliability of the total scale was satisfactory (\(\omega\) = .87); reliability coefficients for each subscale were acceptable: Active coping: \(\omega\) = .89; Emotion-focused coping: \(\omega\) = .77; Avoidance coping: \(\omega\) = 0.82.

\emph{Perceived social support}. The Multidimensional Scale of Perceived Social Support {[}MSPSS; Zimet, Dahlem, Zimet, and Farley (1988){]} was used to evaluate the perceived availability of social support. The MSPSS encompasses three social support subscales, namely family, friends, and significant others, with the items encompassing expressions such as ``I can talk about my problems with my family,'' ``I can count on my friends when things go wrong,'' and ``There is a special person who is around when I am in need.'' Participants were requested to rate their responses to the 12 items on a seven-point Likert scale, with higher total scores indicating greater perceived social support, ranging from ``very strongly disagree'' to ``very strongly agree.'' Previous research has established the MSPSS's good test-retest reliability and discriminant and construct validity (Zimet et al., 1988). For this investigation, we utilized the Italian version of the scale (Prezza \& Principato, 2002). The internal consistency of the current sample was found to be good, with coefficients of \(\omega\) of 0.94 for the family subscale, 0.96 for the friends subscale, and 0.95 for the significant others subscale.

\emph{Post-traumatic stress}. The Impact of Event Scale - Revised {[}IES-R; Weiss (2007){]} was used to evaluate subjective distress associated with traumatic events. The IES-R is a self-report instrument comprising 22 items, designed to capture the essential features of traumatic stress reactions, including intrusion, avoidance, and persistent hyperarousal. These features correspond to criteria B, C, and D of the DSM-IV diagnosis of posttraumatic stress disorder (PTSD). The IES-R includes sub-scales for each of these domains, and it is commonly used to assess PTSD symptomatology in rescue workers. Previous research has demonstrated that the IES-R has good internal consistency and test-retest stability. Additionally, a study by Craparo, Faraci, Rotondo, and Gori (2013) examined the psychometric properties of the Italian translation of the IES-R and found good concurrent and discriminant validity, as well as good test-retest reliability.
In the present sample, IES-R demonstrated high levels of internal consistency, with a total reliability \(\omega\) = .94. The reliability for each of the sub-scales was also high, with \(\omega\) = .91 for intrusion, \(\omega\) = .82 for avoidance, and \(\omega\) = .87 for hyperarousal.

\emph{Post-traumatic growth}. We employed the Post-Traumatic Growth Inventory {[}PTGI; Tedeschi and Calhoun (1996){]} to examine the potential positive changes following one or more traumatic or stressful events. The PTGI is a self-report inventory composed of 21 items and encompasses five subscales, namely Relating to others, New possibilities, Personal strength, Appreciation of life, and Spiritual change. Previous research has demonstrated that the PTGI has good internal consistency, construct-convergent validity, and discriminant validity (Tedeschi \& Calhoun, 1996). Moreover, the Italian version of the PTGI has been found to have good internal consistency and validity (Prati \& Pietrantoni, 2014). In the present study, the PTGI demonstrated high levels of internal consistency, with a total reliability of \(\omega\) = .95. The reliability for each of the sub-scales was also adequate, with \(\omega\) = .91 for Relating to others, \(\omega\) = .84 for New possibilities, \(\omega\) = .84 for Personal strength, \(\omega\) = .79 for Appreciation of life, and \(\omega\) = .75 for Spiritual changes.

\hypertarget{sample-size}{%
\subsection{Sample size}\label{sample-size}}

Based on the Monte Carlo simulation study conducted by Nylund, Asparouhov, and Muthén (2007), a minimum sample size of approximately 500 participants is recommended for accurately identifying the correct number of latent profiles. In previous studies on SC using LPA, D. Y. Liu and Thompson (2017) utilized a sample of 533 participants, and Ullrich-French and Cox (2020) employed three samples with participant sizes of 419, 384, and 509. Hence, our goal was to collect a sample of at least 500 participants. Ultimately, our final sample consisted of 751 RWs.

\hypertarget{participants}{%
\subsection{Participants}\label{participants}}

\hypertarget{material}{%
\subsection{Material}\label{material}}

\hypertarget{procedure}{%
\subsection{Procedure}\label{procedure}}

\hypertarget{data-analysis}{%
\subsection{Data analysis}\label{data-analysis}}

\hypertarget{statistical-analyses}{%
\subsection{Statistical analyses}\label{statistical-analyses}}

Latent Profile Analysis (LPA) is a person-centered latent modeling approach that partitions individuals into discrete classes based on their responses to observed variables. This technique is particularly useful for identifying subgroups of individuals that can be meaningfully compared (Lanza \& Rhoades, 2013). The primary objectives of LVA are twofold. Firstly, to ensure homogeneity within each identified profile so that individuals grouped together are as similar as possible. Secondly, to maximize heterogeneity between profiles so that each profile accurately represents a distinct grouping of individuals. The classes generated by LVA are considered latent since they are not directly observable but are inferred based on similarities in the data. LPA accounts for measurement errors related to the uncertainty in profile membership and provides fit statistics to determine the number of profiles that best represent the data.

The purpose of the LPA was to detect distinct subgroups of RWs who have different profiles on personality dimensions, protective factors, and outcome variables. Standardized scores for five personality measures (neuroticism, extraversion, openness, agreeableness, conscientiousness), three dimensions of coping (COPE-active coping, COPE-avoidance coping, COPE-social emotional coping), perceived social support (MSPSS), and post-traumatic stress (measured using the IES-R) from the RWs were used as observed indicators for the LPA.

We fitted a series of LPA models, ranging from 1 to 10 profiles, using 1000 sets of starting values. To determine the optimal number of profiles, we used information criteria, including Bayesian information criterion (BIC), Akaike information criterion (AIC), and adjusted BIC. We selected the model with the lowest value of these criteria, indicating a better fit. Additionally, we evaluated the accuracy of the classification of individuals into the appropriate profile using entropy, with values closer to 1 indicating higher separation among classes (\textgreater{} 0.80 represents high separation). We also employed the Lo-Mendell-Rubin likelihood ratio test (LMR-LRT), a test statistic to compare the fit of a model with a lower versus higher number of profiles. We used MPLUS 8.6 and the \texttt{R} software for all statistical analyses.

\hypertarget{results}{%
\section{Results}\label{results}}

The use of psychometrically reliable scales allowed for the emergence of the conceptually expected factor structure among the variables that were used used in the study to form the latent profiles (ESEM model with 4 factors, \(\chi^2_{41}\) = 207.52, CFI = 0.95, TLI = 0.89, RMSEA = 0.074, SRMR = 0.034 -- see SI).

\hypertarget{discussion}{%
\section{Discussion}\label{discussion}}

Identifying different profiles of personality factors and coping strategies among rescue workers (RWs) and examining their relationship with self-compassion can provide valuable information for planning treatments aimed at increasing the well-being of RWs. Here's the rationale behind this approach:

\begin{enumerate}
\def\labelenumi{\arabic{enumi}.}
\item
  Tailored Interventions: Understanding the individual differences in personality factors and coping strategies can help in developing targeted interventions. By identifying specific profiles, we can design interventions that address the unique needs and challenges of different groups of RWs. For example, if a lower-resilience profile is associated with higher levels of neuroticism and maladaptive coping strategies, interventions can focus on building emotional stability and promoting adaptive coping skills in this group.
\item
  Enhancing Self-Compassion: Self-compassion has been recognized as a potential protective factor against psychopathology and has demonstrated positive effects on mental health. By investigating the relationship between resilience profiles and self-compassion among RWs, we can identify factors that promote or hinder the development of self-compassion. This knowledge can inform interventions that specifically target self-compassion, aiming to enhance self-care, reduce self-criticism, and foster a compassionate mindset among RWs.
\item
  Prevention of Negative Outcomes: High levels of stress, trauma exposure, and demanding work conditions put RWs at risk of negative outcomes such as compassion fatigue and burnout. By identifying profiles associated with higher vulnerability to these negative outcomes, interventions can be designed to provide targeted support and resources to mitigate their impact. For instance, if a certain profile is found to have low levels of perceived social support, interventions can focus on strengthening social networks and enhancing support systems to prevent burnout.
\item
  Promoting Well-being and Resilience: Ultimately, the goal of these treatments is to increase the overall well-being and resilience of RWs. By understanding the complex interplay between personality factors, coping strategies, and self-compassion, interventions can be tailored to strengthen protective factors, address risk factors, and promote resilience. This can contribute to improving the mental health, job satisfaction, and overall quality of life of RWs, enabling them to effectively cope with the challenges they face in their demanding roles.
\end{enumerate}

By systematically investigating these factors and their relationships, we can gain valuable insights that inform evidence-based interventions, optimize support systems, and contribute to the well-being and resilience of RWs.

Thirdly, this study will explore the impact of direct vs.~indirect involvement (job qualification such as driver, team member, and team leader) in alleviating others' suffering on the reliance on self-compassion. We expected that

The construct of self-compassion is commonly assessed using the Self-Compassion Scale (SCS) self-report questionnaire. The SCS measures six dimensions of self-compassion, three of which evaluate the active components of self-compassion. These dimensions include Self-kindness (SK), Common humanity (CH), and Mindfulness (MI), which involve being kind and understanding towards oneself, recognizing that personal failures and pain are common experiences, and maintaining awareness of one's painful thoughts and feelings. The remaining three dimensions evaluate the ``hindrances'' to self-compassion, including Self-judgment (SJ), Isolation (IS), and Overidentification (OI). These dimensions assess factors that hinder self-compassion, such as being self-critical and unsympathetic towards one's shortcomings, isolating oneself from others, and over-identifying with one's painful thoughts and emotions (Neff, 2022).

Compassion fatigue often leads to a sense of helplessness and a feeling of being unable to do more to help others (Boyle, 2015). This ``learned helplessness'' may lead individuals who experience compassion fatigue to rely more on non-self-compassionate coping strategies compared to those who are less affected by compassion fatigue (Gonzalez-Mendez \& Díaz, 2021).

decrease the use of self-compassionate responding (i.e., self-kindness, common humanity, mindfulness) and increase uncompassionate responding (i.e., self-judgment, isolation, overidentification) -- see also Gonzalez-Mendez and Díaz (2021).

We propose that individuals who experience compassion fatigue may not experience changes in compassionate responding, as their motivation to help others remains intact. However, they may experience a strong increase in uncompassionate responding (e.g., self-blaming), especially in those who are less directly involved in alleviating the suffering of others.

enabling an individual to mobilise internal resources and build up resistance against stressors

individuals with a strong SOC might experience reduced job stress.

\begin{center}\rule{0.5\linewidth}{0.5pt}\end{center}

Sense of coherence (SOC) originates from Antonovsky's (1979,
1987, 1991, 1993) theory of salutogenesis, a paradigm that focuses
on factors that promote health and well-being and considers the
salutary potential of stressors. SOC is a dispositional orientation that
reflects an individual's capacity to cope with life stressors and
comprises three components: Comprehensibility, the sense that
19
stimuli are predictable and structured (cognitive component);
Manageability, the sense that available resources (both internal and
external) are sufficient to cope with demands from the stimuli
(instrumental/behavioural component); and Meaningfulness, the
sense that the demands have significance and are worthy of
investment in terms of personal ideals and standards (motivational
component). An individual's SOC is reinforced by their ``general
resistance resources'' (e.g., intelligence, social support, coping
strategies, and preventative health orientation), which are shaped by
life experiences.

These results indicate the need for improving pre-employment strategies to select the most resilient individuals for rescue work, to implement continuous preventive measures for personnel.

\newpage

\hypertarget{references}{%
\section{References}\label{references}}

\hypertarget{refs}{}
\begin{CSLReferences}{1}{0}
\leavevmode\vadjust pre{\hypertarget{ref-angelini2023big}{}}%
Angelini, G. (2023). Big five model personality traits and job burnout: A systematic literature review. \emph{BMC Psychology}, \emph{11}(1), 1--35.

\leavevmode\vadjust pre{\hypertarget{ref-argentero2011engagement}{}}%
Argentero, P., \& Setti, I. (2011). Engagement and vicarious traumatization in rescue workers. \emph{International Archives of Occupational and Environmental Health}, \emph{84}, 67--75.

\leavevmode\vadjust pre{\hypertarget{ref-beck1990beck}{}}%
Beck, A. T., Steer, R. A., Epstein, N., \& Brown, G. (1990). Beck self-concept test. \emph{Psychological Assessment: A Journal of Consulting and Clinical Psychology}, \emph{2}(2), 191--197.

\leavevmode\vadjust pre{\hypertarget{ref-bianchi2018burnout}{}}%
Bianchi, R. (2018). Burnout is more strongly linked to neuroticism than to work-contextualized factors. \emph{Psychiatry Research}, \emph{270}, 901--905.

\leavevmode\vadjust pre{\hypertarget{ref-bienvenu2004anxiety}{}}%
Bienvenu, O. J., Samuels, J. F., Costa, P. T., Reti, I. M., Eaton, W. W., \& Nestadt, G. (2004). Anxiety and depressive disorders and the five-factor model of personality: A higher-and lower-order personality trait investigation in a community sample. \emph{Depression and Anxiety}, \emph{20}(2), 92--97.

\leavevmode\vadjust pre{\hypertarget{ref-boyle2015compassion}{}}%
Boyle, D. A. (2015). Compassion fatigue: The cost of caring. \emph{Nursing2022}, \emph{45}(7), 48--51.

\leavevmode\vadjust pre{\hypertarget{ref-burton2021individualism}{}}%
Burton, L., Delvecchio, E., Germani, A., \& Mazzeschi, C. (2021). Individualism/collectivism and personality in italian and american groups. \emph{Current Psychology}, \emph{40}, 29--34.

\leavevmode\vadjust pre{\hypertarget{ref-caprara2001brand}{}}%
Caprara, G. V., Barbaranelli, C., \& Guido, G. (2001). Brand personality: How to make the metaphor fit? \emph{Journal of Economic Psychology}, \emph{22}(3), 377--395.

\leavevmode\vadjust pre{\hypertarget{ref-carver1989assessing}{}}%
Carver, C. S., Scheier, M. F., \& Weintraub, J. K. (1989). Assessing coping strategies: A theoretically based approach. \emph{Journal of Personality and Social Psychology}, \emph{56}(2), 267--283.

\leavevmode\vadjust pre{\hypertarget{ref-chatzea2018ptsd}{}}%
Chatzea, V.-E., Sifaki-Pistolla, D., Vlachaki, S.-A., Melidoniotis, E., \& Pistolla, G. (2018). PTSD, burnout and well-being among rescue workers: Seeking to understand the impact of the european refugee crisis on rescuers. \emph{Psychiatry Research}, \emph{262}, 446--451.

\leavevmode\vadjust pre{\hypertarget{ref-connor2007relations}{}}%
Connor-Smith, J. K., \& Flachsbart, C. (2007). Relations between personality and coping: A meta-analysis. \emph{Journal of Personality and Social Psychology}, \emph{93}(6), 1080.

\leavevmode\vadjust pre{\hypertarget{ref-costa1992normal}{}}%
Costa, P. T., \& McCrae, R. R. (1992). Normal personality assessment in clinical practice: The NEO personality inventory. \emph{Psychological Assessment}, \emph{4}(1), 5--13.

\leavevmode\vadjust pre{\hypertarget{ref-craparo2013impact}{}}%
Craparo, G., Faraci, P., Rotondo, G., \& Gori, A. (2013). The impact of event scale--revised: Psychometric properties of the italian version in a sample of flood victims. \emph{Neuropsychiatric Disease and Treatment}, \emph{9}, 1427--1432.

\leavevmode\vadjust pre{\hypertarget{ref-figley1995compassion}{}}%
Figley, C. R. (1995). \emph{Compassion fatigue: Coping with secondary traumatic stress disorder in those who treat the traumatized}. Psychology Press.

\leavevmode\vadjust pre{\hypertarget{ref-gonzalez2021volunteers}{}}%
Gonzalez-Mendez, R., \& Díaz, M. (2021). Volunteers' compassion fatigue, compassion satisfaction, and post-traumatic growth during the SARS-CoV-2 lockdown in spain: Self-compassion and self-determination as predictors. \emph{Plos One}, \emph{16}(9), e0256854.

\leavevmode\vadjust pre{\hypertarget{ref-hashem2020self}{}}%
Hashem, Z., \& Zeinoun, P. (2020). Self-compassion explains less burnout among healthcare professionals. \emph{Mindfulness}, \emph{11}, 2542--2551.

\leavevmode\vadjust pre{\hypertarget{ref-joinson1992coping}{}}%
Joinson, C. (1992). Coping with compassion fatigue. \emph{Nursing}, \emph{22}(4), 116--118.

\leavevmode\vadjust pre{\hypertarget{ref-karsten2012state}{}}%
Karsten, J., Penninx, B. W., Riese, H., Ormel, J., Nolen, W. A., \& Hartman, C. A. (2012). The state effect of depressive and anxiety disorders on big five personality traits. \emph{Journal of Psychiatric Research}, \emph{46}(5), 644--650.

\leavevmode\vadjust pre{\hypertarget{ref-kokkinos2007job}{}}%
Kokkinos, C. M. (2007). Job stressors, personality and burnout in primary school teachers. \emph{British Journal of Educational Psychology}, \emph{77}(1), 229--243.

\leavevmode\vadjust pre{\hypertarget{ref-kotov2010linking}{}}%
Kotov, R., Gamez, W., Schmidt, F., \& Watson, D. (2010). Linking {``big''} personality traits to anxiety, depressive, and substance use disorders: A meta-analysis. \emph{Psychological Bulletin}, \emph{136}(5), 768--821.

\leavevmode\vadjust pre{\hypertarget{ref-lanza2013latent}{}}%
Lanza, S. T., \& Rhoades, B. L. (2013). Latent class analysis: An alternative perspective on subgroup analysis in prevention and treatment. \emph{Prevention Science}, \emph{14}(2), 157--168.

\leavevmode\vadjust pre{\hypertarget{ref-liu2021self}{}}%
Liu, A., Wang, W., \& Wu, X. (2021). Self-compassion and posttraumatic growth mediate the relations between social support, prosocial behavior, and antisocial behavior among adolescents after the ya'an earthquake. \emph{European Journal of Psychotraumatology}, \emph{12}(1), 1--12.

\leavevmode\vadjust pre{\hypertarget{ref-liu2017selection}{}}%
Liu, D. Y., \& Thompson, R. J. (2017). Selection and implementation of emotion regulation strategies in major depressive disorder: An integrative review. \emph{Clinical Psychology Review}, \emph{57}, 183--194.

\leavevmode\vadjust pre{\hypertarget{ref-lyne2000psychometric}{}}%
Lyne, K., \& Roger, D. (2000). A psychometric re-assessment of the COPE questionnaire. \emph{Personality and Individual Differences}, \emph{29}(2), 321--335.

\leavevmode\vadjust pre{\hypertarget{ref-macbeth2012exploring}{}}%
MacBeth, A., \& Gumley, A. (2012). Exploring compassion: A meta-analysis of the association between self-compassion and psychopathology. \emph{Clinical Psychology Review}, \emph{32}(6), 545--552.

\leavevmode\vadjust pre{\hypertarget{ref-malouff2005relationship}{}}%
Malouff, J. M., Thorsteinsson, E. B., \& Schutte, N. S. (2005). The relationship between the five-factor model of personality and symptoms of clinical disorders: A meta-analysis. \emph{Journal of Psychopathology and Behavioral Assessment}, \emph{27}, 101--114.

\leavevmode\vadjust pre{\hypertarget{ref-mao2022concept}{}}%
Mao, X., Hu, X., \& Loke, A. Y. (2022). A concept analysis on disaster resilience in rescue workers: The psychological perspective. \emph{Disaster Medicine and Public Health Preparedness}, \emph{16}(4), 1682--1691.

\leavevmode\vadjust pre{\hypertarget{ref-mccann1990vicarious}{}}%
McCann, I. L., \& Pearlman, L. A. (1990). Vicarious traumatization: A framework for understanding the psychological effects of working with victims. \emph{Journal of Traumatic Stress}, \emph{3}, 131--149.

\leavevmode\vadjust pre{\hypertarget{ref-murray2003neo}{}}%
Murray, G., Rawlings, D., Allen, N. B., \& Trinder, J. (2003). NEO five-factor inventory scores: Psychometric properties in a community sample. \emph{Measurement and Evaluation in Counseling and Development}, \emph{36}(3), 140--149.

\leavevmode\vadjust pre{\hypertarget{ref-neff2003self}{}}%
Neff, K. D. (2003). Self-compassion: An alternative conceptualization of a healthy attitude toward oneself. \emph{Self and Identity}, \emph{2}(2), 85--101.

\leavevmode\vadjust pre{\hypertarget{ref-neff2022differential}{}}%
Neff, K. D. (2022). The differential effects fallacy in the study of self-compassion: Misunderstanding the nature of bipolar continuums. \emph{Mindfulness}, \emph{13}(3), 572--576.

\leavevmode\vadjust pre{\hypertarget{ref-nylund2007deciding}{}}%
Nylund, K. L., Asparouhov, T., \& Muthén, B. O. (2007). Deciding on the number of classes in latent class analysis and growth mixture modeling: A monte carlo simulation study. \emph{Structural Equation Modeling: A Multidisciplinary Journal}, \emph{14}(4), 535--569.

\leavevmode\vadjust pre{\hypertarget{ref-prati2014italian}{}}%
Prati, G., \& Pietrantoni, L. (2014). Italian adaptation and confirmatory factor analysis of the full and the short form of the posttraumatic growth inventory. \emph{Journal of Loss and Trauma}, \emph{19}(1), 12--22.

\leavevmode\vadjust pre{\hypertarget{ref-prezza2002rete}{}}%
Prezza, M., \& Principato, M. C. (2002). La rete sociale e il sostegno sociale. \emph{Conoscere La Comunit{à}}, 193--233.

\leavevmode\vadjust pre{\hypertarget{ref-ruaducu2022personality}{}}%
Răducu, C.-M., \& Stănculescu, E. (2022). Personality and socio-demographic variables in teacher burnout during the COVID-19 pandemic: A latent profile analysis. \emph{Scientific Reports}, \emph{12}(1), 14272.

\leavevmode\vadjust pre{\hypertarget{ref-sica2021facing}{}}%
Sica, C., Latzman, R. D., Caudek, C., Cerea, S., Colpizzi, I., Caruso, M., \ldots{} Bottesi, G. (2021). Facing distress in coronavirus era: The role of maladaptive personality traits and coping strategies. \emph{Personality and Individual Differences}, \emph{177}, 110833.

\leavevmode\vadjust pre{\hypertarget{ref-sica2008coping}{}}%
Sica, C., Magni, C., Ghisi, M., Altoè, G., Sighinolfi, C., Chiri, L. R., \& Franceschini, S. (2008). Coping orientation to problems experienced-nuova versione italiana (COPE-NVI): Uno strumento per la misura degli stili di coping. \emph{Psicoterapia Cognitiva e Comportamentale}, \emph{14}(1), 27.

\leavevmode\vadjust pre{\hypertarget{ref-sica1997coping}{}}%
Sica, C., Novara, C., Dorz, S., \& Sanavio, E. (1997). Coping strategies: Evidence for cross-cultural differences? A preliminary study with the italian version of coping orientations to problems experienced (COPE). \emph{Personality and Individual Differences}, \emph{23}(6), 1025--1029.

\leavevmode\vadjust pre{\hypertarget{ref-swider2010born}{}}%
Swider, B. W., \& Zimmerman, R. D. (2010). Born to burnout: A meta-analytic path model of personality, job burnout, and work outcomes. \emph{Journal of Vocational Behavior}, \emph{76}(3), 487--506.

\leavevmode\vadjust pre{\hypertarget{ref-tahernejad2023post}{}}%
Tahernejad, S., Ghaffari, S., Ariza-Montes, A., Wesemann, U., Farahmandnia, H., \& Sahebi, A. (2023). Post-traumatic stress disorder in medical workers involved in earthquake response: A systematic review and meta-analysis. \emph{Heliyon}, e12794.

\leavevmode\vadjust pre{\hypertarget{ref-tedeschi1996posttraumatic}{}}%
Tedeschi, R. G., \& Calhoun, L. G. (1996). The posttraumatic growth inventory: Measuring the positive legacy of trauma. \emph{Journal of Traumatic Stress}, \emph{9}(3), 455--471.

\leavevmode\vadjust pre{\hypertarget{ref-ullrich2020use}{}}%
Ullrich-French, S., \& Cox, A. E. (2020). The use of latent profiles to explore the multi-dimensionality of self-compassion. \emph{Mindfulness}, \emph{11}, 1483--1499.

\leavevmode\vadjust pre{\hypertarget{ref-veneziani2017self}{}}%
Veneziani, C. A., Fuochi, G., \& Voci, A. (2017). Self-compassion as a healthy attitude toward the self: Factorial and construct validity in an italian sample. \emph{Personality and Individual Differences}, \emph{119}, 60--68.

\leavevmode\vadjust pre{\hypertarget{ref-weiss2007impact}{}}%
Weiss, D. S. (2007). The impact of event scale: revised. In \emph{Cross-cultural assessment of psychological trauma and PTSD} (pp. 219--238). Springer.

\leavevmode\vadjust pre{\hypertarget{ref-wilson2019effectiveness}{}}%
Wilson, A. C., Mackintosh, K., Power, K., \& Chan, S. W. (2019). Effectiveness of self-compassion related therapies: A systematic review and meta-analysis. \emph{Mindfulness}, \emph{10}(6), 979--995.

\leavevmode\vadjust pre{\hypertarget{ref-wong2017self}{}}%
Wong, C. C. Y., \& Yeung, N. C. (2017). Self-compassion and posttraumatic growth: Cognitive processes as mediators. \emph{Mindfulness}, \emph{8}(4), 1078--1087.

\leavevmode\vadjust pre{\hypertarget{ref-zimet1988multidimensional}{}}%
Zimet, G. D., Dahlem, N. W., Zimet, S. G., \& Farley, G. K. (1988). The multidimensional scale of perceived social support. \emph{Journal of Personality Assessment}, \emph{52}(1), 30--41.

\end{CSLReferences}


\end{document}
